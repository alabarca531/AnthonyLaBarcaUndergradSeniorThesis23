\chapter{Introduction}
\label{chap:introduction}
%%%%%%%%%%%%%%%%%%%%%%%%%%%%%%%%%%%%%%%%%%%%%%%%%%%%%%%%%%%%%%%
% OUTLINE FOR INTRODUCTION 
% 1. Quick introduction of topic
    % 1.1. Traditional methods of identifying probelmatic data
    % 1.2. Machine Learning applied to more and more mundane tasks
% 2. Problem statement
    % 2.1. What is the problem?
    % 2.2. Why is it important?
% 3. Scope
    % 3.1. What is the current landscape? 
    % 3.2. How does this fit into that landscape
% 4. Objectives
    % 3.1. What does this add to the current landscape?
% 5. Structure of the thesis
    % 5.1. What is the structure of the thesis?
%%%%%%%%%%%%%%%%%%%%%%%%%%%%%%%%%%%%%%%%%%%%%%%%%%%%%%%%%%%%%%%

\section{Citation styles}

These are the different citation styles for author-year.

\noindent The standard \verb=\cite= command produces the following output: \cite{cybenko1989}.

\noindent The \verb=\textcite= command produces the following output: \textcite{cybenko1989}.


\noindent The \verb=\parencite= command produces the following output: \parencite{cybenko1989}.

\noindent The \verb=\footcite= command produces a footnote citation \footcite{cybenko1989}.


