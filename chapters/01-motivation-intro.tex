\chapter{Introduction}
\label{chap:introduction}
%%%%%%%%%%%%%%%%%%%%%%%%%%%%%%%%%%%%%%%%%%%%%%%%%%%%%%%%%%%%%%%
% OUTLINE FOR INTRODUCTION 
% 1. Quick introduction of topic
    % 1.1. Traditional methods of identifying probelmatic data
    % 1.2. Machine Learning applied to more and more mundane tasks
% 2. Problem statement
    % 2.1. What is the problem?
    % 2.2. Why is it important?
% 3. Scope
    % 3.1. What is the current landscape? 
    % 3.2. How does this fit into that landscape
% 4. Objectives
    % 3.1. What does this add to the current landscape?
% 5. Structure of the thesis
    % 5.1. What is the structure of the thesis?
%%%%%%%%%%%%%%%%%%%%%%%%%%%%%%%%%%%%%%%%%%%%%%%%%%%%%%%%%%%%%%%

%% Segev Outline for Introduction
% 
% SN classification & Identification: Why important?
% Spectra as "gold standard" for classifiers 
% Attempt at spectroscopic classification
% Madgwich et al, Gravr et al (2013), Bovan et al, Mathukrishna et al (2019)
% AI/ ML approach: why better, possibly? 
% What problems does it solve
% what will you be looking at 
% i.e. de-redshifted vs not, subtypeclassification vs none, etc.
% "We will explore the following questions:

When observing the night sky, astronomers frequently encounter objects that 
were unexpected. This occured quite frequently in the 19th century, as astronomers 
armed with more powerful telescopes were probing the cosmos for the first time. 
In 1885 and 1895, two objects were observed in galaxies outside of our own with unique 
properties \parencite{deVaucouleurs1985, Schaefer1995}. These objects were very 
bright, outshining their host galaxies, and faded away over the course of a few months.
These objects were eventually named `supernovae' (SNe) by Walter Baade and Fritz Zwicky, who were the first to suggest that
these objects were the result of the death of a star, and predict the resulting 
object to be a cold, neutron star\parencite{Baade1934}. Since then, astronomers 
have observed thousands of SNe~\parencite{Barbon1999}, and have
developed a classification scheme for these objects.  

\section{Supernova Classification}
\label{sec:supernova-classification}
SNe have been traditionally classified based on their spectra since 
\textcite{Minkowski1941} first classified SNe into two groups based on the
presence or absence of hydrogen in their spectra. This classification provided 
the basis for the current classification scheme, which is based on the presence
or absence of hydrogen and silicon in the spectra of SNe. This classification 
scheme is shown in Figure~\ref{fig:sn-classification}. 

\begin{figure}[ht]
    \centering
    \includegraphics[width=0.8\linewidth]{images/sn-classification.png}
    \caption[Supernova Classification]{Supernova Classification. The 
    classification scheme is based on the presence or absence of hydrogen and 
    silicon in the spectra of SNe. Figure from \textcite{Turatto2003}.}
    \label{fig:sn-classification}
\end{figure}

The classification scheme in Figure~\ref{fig:sn-classification} is based on
the presence or absence of hydrogen and silicon in the spectra of SNe.
SNe with hydrogen in their spectra are classified as Type II SNe, while those
without hydrogen are classified as Type I SNe. Type I SNe are further
subdivided into Type Ia, Ib, and Ic~\parencite{Turatto2003}. 

Although divided into three subtypes, Type I SNe are caused by two 
different mechanisms. Type Ia SNe are caused by the thermonuclear explosion
of a white dwarf star, while Type Ib and Ic SNe are caused by the core-collapse
of a massive star~\parencite{Filippenko1997}. Type Ib and Ic SNe, because of this, 
are related to Type II SNe, and are collectively referred to as core-collapse
SNe. Stars stripped of their hydrogen envelope are classified as Type Ib SNe, 
while those stripped of both their hydrogen and helium envelopes are classified
as Type Ic SNe. 

\section{Supernova Identification}
\label{sec:supernova-identification}
The different types of SNe are caused by different mechanisms, and therefore 
knowing the type of a SN can provide insight into the mechanism that caused it.
Detecting an SNe is therefore not enough, and astronomers must also identify the
type of the SN. Traditionally, the supernova would be observed by alternative 
means, followed up by photometric and spectroscopic observations, as presented by 
\textcite{Perlmutter1999}. This process is time-consuming, requiring not only 
the initial discovery of the SN, but also follow-up observations to determine
the type of the SN before it fades away. 

\section{Dark Energy Spectroscopic Instrument}
\label{sec: DESI}
* Write a section about DESI
* What is it?
* Why is it important?
* Why do we care about SNe here

\section{Supernova Classification with Machine Learning}
\label{sec:supernova-classification-with-machine-learning}
Machine learning (ML) has been applied to many different fields for its flexibility 
and ability to find patterns in data. The emergence of Deep Learning (DL) has
allowed ML to be applied to more and more mundane tasks, such as image
classification~\parencite{krizhevsky2012}, speech recognition~\parencite{Nassif2019},
and natural language processing~\parencite{Mikolov2013}. ML and DL 
has also been applied to astronomy, with some success. \textcite{Gauci2010} used 
a random forest algorithm  to distinguish between spiral, elliptical, and irregular galaxies, 
and \textcite{Becker2021} used a convolutional neural network (CNN) to classify the morphology of
radio galaxies. CNNs found success in working within the DESI collaboration, with 
\textcite{parks2018} using the architecture to detect strong emission lines.  
These techniques have also been applied to the classification of SNe. 
\textcite{Mller2016} used a CNN to classify SNe into Type Ia and non-Type Ia SNe. 

\section{Problem Statement}
This work will explore the following questions: how can we adapt novel deep learning 
techniques to classify SNe in the DESI dataset? How can we use this architecture to 
classify SNe into their respective subtypes accurately? And how can we decrease the amount of 
pre-processing required to classify these SNe accurately? 

