\chapter{Introduction}
\label{chap:introduction}
%%%%%%%%%%%%%%%%%%%%%%%%%%%%%%%%%%%%%%%%%%%%%%%%%%%%%%%%%%%%%%%
% OUTLINE FOR INTRODUCTION 
% 1. Quick introduction of topic
    % 1.1. Traditional methods of identifying probelmatic data
    % 1.2. Machine Learning applied to more and more mundane tasks
% 2. Problem statement
    % 2.1. What is the problem?
    % 2.2. Why is it important?
% 3. Scope
    % 3.1. What is the current landscape? 
    % 3.2. How does this fit into that landscape
% 4. Objectives
    % 3.1. What does this add to the current landscape?
% 5. Structure of the thesis
    % 5.1. What is the structure of the thesis?
%%%%%%%%%%%%%%%%%%%%%%%%%%%%%%%%%%%%%%%%%%%%%%%%%%%%%%%%%%%%%%%

\section{Citation styles}

These are the different citation styles for author-year.

The standard \verb=\cite= command produces the following output: \cite{clarke1990rendezvous}.

The \verb=\textcite= command produces the following output: \textcite{clarke1990rendezvous}.


The \verb=\parencite= command produces the following output: \parencite{clarke1990rendezvous}.

The \verb=\footcite= command produces a footnote citation \footcite{clarke1990rendezvous}.

\clearpage          % To start a new page

\section{Figures}

Always prefer to float figures to the top of the page using the [t] option in the figure environment. This is 
shown in Figure \ref{fig:fig_1}.

\begin{figure}[t]
    \centering
    \includegraphics[width=.4\textwidth]{figures/blackbox.jpeg}
    \caption{This is a black box}
    \label{fig:fig_1}
\end{figure}

\clearpage

\section{Tables}

Use vertical lines sparingly in tables. They're unnecessary bloat. Write the code for tables in a separate tex file, and include it in when required. 
Also preferrably use sans serif font for tables (because of their information density) using \texttt{\\sffamily} in table definition.

\begin{table}[!ht]
\small
\centering
\sffamily
\input{tables/table.tex}
\caption{This is a table}
\label{tab:table}
\vspace{-5mm}
\end{table}

