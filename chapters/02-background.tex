\chapter{Machine Learning Techniques}
\label{chap:MLTechniques}
%%%%%%%%%%%%%%%%%%%%%%%%%%%%%%%%%%%%%%%%%%%%%%%%%%%%%%%%%%%%%%%
% Outline 
% Introduce Machine Learning, and why traiditonal FCN is not good enough
% 1. Convolutional Neural Networks
% 1.1 Example architecture and sucess in vision tasks
% 1.2 1D CNN implemented on DESI spectra
% 1.3 2D CNN implemented on DESI spectra (Eddies Research)
% 2. Transformers 
% 2.1 Example architecture and sucess in NLP tasks + Vision Tasks --> Connection 
%      to DESI
% 2.2 Multi-Head Attention and Self-Attention for spectral classification
% 2.3 Possilbility of NOT having to k-correct for redshift because of this attention
% 2.4 Adjected ViT architecture for DESI

%%%%%%%%%%%%%%%%%%%%%%%%%%%%%%%%%%%%%%%%%%%%%%%%%%%%%%%%%%%%%%%
To aid in the classification of supernovae, machine learning techniques provide 
flexibility and scalability that have been previously unattainable by manual 
methods. The simplest form of machine learning algorithms are the feed-forward 
fully connected neural networks (FCNs). These networks apply a series of linear 
transformations to the input data, followed by a bias and a non-linear activation 
function. The output of the network is another vector tailored for whatever the 
purpose of the FCN is (i.e. for classification, is a vector of probabilities 
that the input data belongs to, see Appendix~\ref{app:FCN}). 

This architecture, while simplistic, is very powerful. \textcite{cybenko1989} showed that for any boolean 
function (i.e. 2 class classification), a FCN with a single hidden layer could 
approximate any function arbitrarily well; this is known as the universal approximation
theorem. This theorem is the basis for the success of FCNs in
many fields, but also the reason why they are not the best choice for all problems.
The main issue with FCNs is that they are not invariant to translations, which is 
essential for many problems such as image recognition, time series analysis, and 
natural language processing. Therefore, alternative options have been proposed. 
\section{Convolutional Neural Networks}
\label{sec:CNN}
Convolutional neural networks (CNNs) were originally developed as a solution 
to positional differences in the input data \parencite{fukushima1979}. This architecture 
was shown to be effective in image recognition tasks by \textcite{lecun2004}, and 
was popularized after the success of AlexNet in the 
ImageNet challenge \textcite{krizhevsky2012}, and have since been applied for use in 
many other vision tasks, as well as other fields such as NLP and audio processing. 
CNNs are composed of a series of convolutional layers, usually followed by a 
pooling layer, and then a fully connected layer (Appendix~\ref{app:CNN}). The convolutional layers are 
composed of a series of convolutional filters (usually referred to as feature maps)
which are applied across the entire input space. The application of the same filter 
across the entire input space is what gives the CNN its invariance to translations.

\subsection{CNNs on Spectroscopic Data}
\label{sec:CNNspectra}
Due to their affinity for disregarding spacial invariance, CNNs theoretically should 
translate to spectral classification quite well. In fact, CNNs have been applied t
o spectroscopic data in the past, even on the DESI dataset. 
\textcite{parks2018} used a CNN to detect strong emission lines in DESI spectra with 
great success. * Talk about 1D CNN currently running, and preprocessing used *

This work provides a baseline for the use of CNNs on supernovae classification, 
but leaves more to be desired. Therefore, an alternate architecture was proposed 
by * Cite Eddies work * which augments the preprocessing of the spectra with 
a conversion to a 2D image. This 2D image was then fed to a more traditional vision 
CNN architecture (Appendix~\ref{app:CNN}). This change lead to a model that was 
able to determine the 
\section{Previous and Current Attempts at Machine Learning}
\label{sec:previousML}
% talk about advantage of CNN

\section{Introduction of Transformers}
\label{sec:transformers}
% Advantage of transformers in this field

