\chapter{Validity of Transformers for Spectral Classification}
\label{chap:chapter-3}
\section{Creation of artificial spectra}
\label{sec:creation}
Creating large amounts of high signal-to-noise signals was essential to verifying 
the application of a Vision Transformer. These signals needed to include the 
basic features of supernova spectrum, such as the presence of a continuum, 
absorption lines, and emission lines. The locations of these spectral features 
in particular were not important, as long as each class of supernovae had consistent 
features. The \texttt{GenData} class was created to generate these signals. 

\subsection{Features of \texttt{GenData} class}
\label{ssec:features}
The \texttt{GenData} class must provide a means of generating a large number (order of $10^5$)
of random spectra. These spectra must exhibit a consistent continuum, variable noise, 
and a set of spectral features that are unique to arbitrary classes. In order to 
accomplish this, the \texttt{GenData} class must first identify a domain in which 
to place spectral features, noise, and the continuum. The continuum is a function across
the domain that remains consistent for all samples, and examples can be found in Fig.~\ref{fig:continuumoptions}.
A set number of spectral features are randomly placed within the domain. The number of spectral
features is determined by the user. Then, each class (or type) of spectra is assigned 
a random combination of these spectral features. This is to ensure that each class 
has a unique set of spectral features, while maintaining a consistent location of features. 
Once the different classes are specified, the creation of the spectra can begin.

\begin{figure}
    \centering
    \includegraphics[width=0.5\textwidth]{figures/blackbox.jpeg}
    \caption{Examples of different continuum options
    (from left to right: linear, linear increasing, linear decreasing, 
exponential increasing, and exponential decreasing).}
    \label{fig:continuumoptions}
\end{figure}

\subsection{Creation of spectra}
\label{ssec:creation}
Once the spectral features present in each class are determined, the spectra can 
be constructed from the ground up. First, the continuum is created 

% The essential Some of the features of the \texttt{GenData} class are as follows:
% \begin{itemize}
%     \item The ability to generate a large number (order of $10^5$) of random spectra
%     \item The ability to generate spectra with a continuum
%     \item The ability to generate spectra with absorption lines
%     \item The ability to generate spectra with emission lines
%     \item The ability to generate spectra with a combination of all three
%     \item The ability to generate spectra with a combination of all three, but with 
%     different spectral features in each spectrum

% \end{itemize}
