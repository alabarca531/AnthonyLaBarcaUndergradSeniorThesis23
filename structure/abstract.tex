\chapter*{\vspace{-1cm}\centerline{Abstract}\vspace{-3cm}}
\markboth{\MakeUppercase{Abstract}}{}
\iftoggle{fulltoc}{
  \addcontentsline{toc}{chapter}{Abstract}
}{}

The identification and classification of supernovae transients are common hindrances in observational astronomy as they require 
additional spectroscopic observations and traditional autonomous classification 
techniques that are computationally expensive. In an effort to create a more efficient and autonomous 
classification methodology, deep learning architectures have previously been trained and implemented 
for use on the Dark Energy Spectroscopic Instrument (DESI) pipeline. Despite being  able to precisely classify some spectra, the networks are unable to definitively classify many more, leading to 
a recall rate of approximately 20\%. The Spectral ViT is a vision transformer-based architecture
trained for use on the DESI pipeline. When trained on spectra with redshift removed according to the
fits provided by the DESI pipeline, the Spectral ViT (V1) is able to recall 26\% more targets while also applying a 
stricter prediction cut, resulting in only a 13.3\% decrease in performance. The Spectral ViT V2 
was trained for the purpose of eliminating dependence on the DESI redshift fit.  While unable to 
match the sub-type classification power of the Spectral ViT V1, the model shows capability in acting as a 
classifier for progenitor and supernovae type, with precision values of 86.5\% and 91.8\% respectively. We find
the Spectral ViT V1 to be a possible replacement for the existing classification network pending further testing
and the Spectral ViT V2 to be a supernovae filter and supplemental classifier. 